AppEvolve is the state-of-the-art approach for automatic update of
deprecated-API usage in Android apps. Experiments previously reported for
AppEvolve have shown that it can generate applicable updates for
85\% of these API
changes and 90\% of their usage locations on mobile apps in the
AppEvolve dataset.

In this work, we evaluate whether this observed effectiveness is
generalizable. We add 54 additional mobile apps that use the APIs contained
in the AppEvolve dataset. Running AppEvolve on these mobile apps shows that
AppEvolve fails to generate applicable updates for 81\% of the mobile
apps. By analyzing these failed cases, we found that they failed mainly
due to:
\begin{enumerate}
    \item Statements in the examples and at the target location are structurally different.
    \item Object and arguments of the deprecated API method are in the form of complex expressions.
    \item Edits are required beyond method boundaries.
    \item Incomplete support of programming language features.
    \item No examples found in GitHub.
\end{enumerate}
We mitigate the first and the second categories by performing a simple
refactoring that modifies the code containing API usage in the target app
to resemble the one in the example. Our mitigations enable AppEvolve
to generate applicable updates for 82\% of the failed cases.

To try to resolve the discrepancy in the findings, we looked closer at the
examples and target apps used by finding the original examples and target
apps. We found that they were transformed analogous to our mitigation.
These transformations convert invocations in various contexts into code
exploitable by AppEvolve.

Based on our findings, we propose future directions for automatic update of
deprecated-API usage in Android apps, which include code normalization to
ensure that both the example of API usage and the target app code are in
the same standard form to minimize the variety of ways that the code is
written. Using this technique, simple refactoring that we use to mitigate
the failed cases may not be necessary. We also propose the usage of
identifier name recommendation to name new variables that may have been
introduced due to code normalization.
