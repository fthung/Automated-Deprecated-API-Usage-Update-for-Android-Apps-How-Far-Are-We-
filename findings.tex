On our dataset, AppEvolve produces 10 applicable updates and 44 failed
updates (19\% success rate). The original AppEvolve evaluation also showed 4 failed updates.
After investigating the 48 failed updates, we found some common reasons for
the failures, as described below. Table~\ref{table:data_statistic} shows
the number of occurrences of each of these issues.

% \begin{table}
% \caption{Update findings statistic}
% \begin{center}
% \begin{tabular}{ | p{8em} |c|c| }
%  \hline
%  \textbf{Test Case Type} & \textbf{Detail of Cases} & \textbf{Number of Cases} \\
%  \hline
%  Work in AppEvolve & - & 10 \\
%  \hline
%  \multirow{7}{8em}{Fixed with simple refactor} & Total cases & 36 \\\cline{2-3} & Return statement & 6 \\\cline{2-3} & If statement & 1 \\\cline{2-3} & Method argument & 4 \\\cline{2-3} & Arithmetic operand & 3 \\\cline{2-3}  & Variable declaration & 12 \\\cline{2-3} & Complex expression & 20 \\\cline{2-3}
%  \hline
%  \multirow{3}{8em}{Incomplete support of programming language features} & Inheritance & 2\\\cline{2-3} & Static modifier & 1 \\\cline{2-3} & Final modifier & 1\\\cline{2-3}
%  \hline
%  Unknown error & - & 4\\
%  \hline
% \end{tabular}
% \end{center}
% \label{table:data_statistic}
% \end{table}

\begin{table}
\caption{Update findings statistics (some occurrences of deprecated APIs
are in multiple categories)}
\begin{center}
\begin{tabular}{ | p{12em} |c|c| }
 \hline
 \textbf{Category} & \textbf{Subcategory} & \textbf{Occurrences} \\
 %\hline
 %Work in AppEvolve & - & 10 \\
 \hline
 \multirow{5}{12em}{Statements in the examples and at the target location are structurally different} & Return statement & 6 \\\cline{2-3} & If statement & 1 \\\cline{2-3} & Method argument & 4 \\\cline{2-3} & Arithmetic operand & 3 \\\cline{2-3}  & Declared variable & 12 \\\cline{2-3}
 \hline
  \multirow{4}{12em}{Object and arguments of the deprecated API method are in the form of complex expressions} & - & 20\\&&\\&&\\&&\\
 \hline
 Edits beyond method boundaries & - & 3\\
\hline
 \multirow{3}{12em}{Incomplete support of programming-language features} & Inheritance & 2\\\cline{2-3} & Static modifier & 1 \\\cline{2-3} & Final modifier & 1\\\cline{2-3}
 \hline
 No examples & - & 1\\
 \hline
 Others & - & 4\\
 \hline
\end{tabular}
\end{center}
\label{table:data_statistic}
\end{table}

\vspace{0.25\baselineskip}\noindent\textbf{1) The statements in the examples and at the target location are
 structurally different.} AppEvolve infers edit operations at the statement
 level (insert a statement, move a statement, etc).  Accordingly, AppEvolve
 is not able to apply changes inferred about method uses found in one kind
 of statement to a method use found in another kind of statement.  We have
 observed that in the examples used in the evaluation of AppEvolve, it
 often occurs that the invocation of the deprecated API method is used as
 the right hand side of an assignment, while in our dataset the API method
 invocations occur in a variety of expression contexts.  We now present
 some examples:
\begin{itemize}
\item Return statement (left side of Listing 1 in
Table~\ref{tab:mitigatesucc}): An invocation of the deprecated method {\tt
fromHtml} appears as part of a return statement of the {\tt getTitle}
method.

\item If statement test (left side of Listing 2 in
Table~\ref{tab:mitigatesucc}): An invocation of the deprecated method {\tt
request\-Audio\-Focus} appears as a subexpression of the test expression of a
conditional.

\item Method argument (left side of Listing 3 in
Table~\ref{tab:mitigatesucc}): An invocation of the deprecated method {\tt
getCurrentHour} appears as the second argument of the invocation of method
{\tt String.format}.

\item Binary operator (left side of Listing 4 in
Table~\ref{tab:mitigatesucc}): An invocation of the deprecated method {\tt
getCurrentHour} appears as the left argument of a concatenation with {\tt ":"}.

\item Declared variable (left side of Listing 5 in
Table~\ref{tab:mitigatesucc}): An invocation of the deprecated method {\tt
getCurrentHour} appears as the initial value of a declared variable.  Even
though this case involves an assignment, such code does not match examples
where the assignment involves a previously declared variable.
\end{itemize}

\vspace{0.25\baselineskip}\noindent\textbf{2) Object and arguments of the
deprecated API method are complex expressions.}
In creating edits, AppEvolve only abstracts over variables.  Accordingly,
when examples always contain variables for the object or the arguments, the
generated edits are not sufficient to update code in which these subterms
are expressed as more complex expressions.

In the code shown on the left side of Listing 6 in
Table~\ref{tab:mitigatesucc}, an invocation of the deprecated method {\tt
setTextAppearance} involves expressions that do not have the form
of a simple variable for both the object and arguments.

\vspace{0.25\baselineskip}\noindent\textbf{3) Edits beyond method boundaries.} AppEvolve cannot learn edits that modify program elements that reside outside of the method containing the API usage to be updated. These edits include operations such as adding imports and adding fields to a class. Below is a snippet of an update example that involves edits beyond method boundaries.
\begin{lstlisting}[language=diff,numbers=none]
18a19
+ import android.location.GnssStatus;
22a24
+ import android.os.Build;
33a36,37
+ private GnssStatus.Callback callback;
41c45,51
- locationManager.addGpsStatusListener(listener);
+ if (android.os.Build.VERSION.SDK_INT >=
            android.os.Build.VERSION_CODES.N) {
+   locationManager.registerGnssStatusCallback(callback);
+ }
+ else{
+   locationManager.addGpsStatusListener(listener);
+ }
\end{lstlisting}
The above example involves a use of the deprecated API {\tt
addGpsStatusListener}. Updating a use of this API requires adding a private
field {\tt callback} of type {\em GnssStatus.Callback}.
Since AppEvolve only learns edits inside a method containing the deprecated
API usage, it misses this necessary addition.  In the AppEvolve paper, this
category is described as missing context information.

\vspace{0.25\baselineskip}\noindent\textbf{4) Incomplete support of programming language features.} AppEvolve fails to update cases in which the update involves programming language features such as:
\begin{itemize}
\item Inheritance: The code on the left hand side of Listing 7 in
Table~\ref{tab:mitigatefail} uses the deprecated API method {\tt
set\-Audio\-Stream\-Type} of the class {\tt MediaPlayer}. In the code snippet,
{\tt Test\-Media\-Player} extends {\tt MediaPlayer} and thus it inherits
the {\tt setAudioStreamType} method. This method should be updated but
AppEvolve does not seem to recognize it due to the use of inheritance.
\item Static modifier:  The code on the left hand side of Listing 8 in
Table~\ref{tab:mitigatefail} shows an example of this case.  The deprecated
API method {\tt requestAudioFocus} is invoked by {\tt mAudioManager}, which
is a static field.

\item Final modifier: The code in the left hand side of Listing 9 in
Table~\ref{tab:mitigatefail} shows an example of this case.  Variable {\tt
paint} that is used as the fourth argument to the deprecated API method
{\tt saveLayer} is a final field.

\end{itemize}

\vspace{0.25\baselineskip}\noindent\textbf{5) No examples.} No example
illustrating the API update can be found in GitHub.

\vspace{0.25\baselineskip}\noindent\textbf{6) Others.} These include cases that cannot be put to any category above.

\vspace{0.25\baselineskip}\noindent The categories {\em edits
beyond method boundaries} and {\em no examples} are from target apps in the
original AppEvolve dataset and these categories are reported in the
AppEvolve paper. The other categories are uncovered from our dataset.

\newcolumntype{Y}{>{\centering\arraybackslash}X}

\lstset{
	language=text,numbers=none, breaklines=true, aboveskip=-7pt,
	belowskip= -6pt }

\begin{table*}
\centering
\caption{Successful simple refactoring mitigations that allow AppEvolve to generate applicable updates}\label{tab:mitigatesucc}
\begin{tabular}{|p{.10\textwidth}|p{.40\textwidth}|p{.40\textwidth}|}
\hline
\textbf{Listing}
  &
  \textbf{Original Code}
  &
  \textbf{Refactoring Diff}
 \\ \hline
1. Structurally \newline different: return statement
&
\begin{lstlisting}
Spanned getTitle(){
    return Html.fromHtml(title);
}
\end{lstlisting}
&
\begin{lstlisting}[language=diff]
Spanned getTitle(){
+   Spanned a;
+   a = Html.fromHtml(title);
-   return Html.fromHtml(title);
+   return a;
}
\end{lstlisting}
\\ \hline
2. Structurally \newline different: if \newline statement
&
\begin{lstlisting}
public void play() {
    if (mAudioManager.requestAudioFocus(
            mAudioFocusListener,
            AudioManager.STREAM_MUSIC,
            AudioManager.AUDIOFOCUS_GAIN) !=
        AudioManager.AUDIOFOCUS_REQUEST_GRANTED){
        return;
    }
}
\end{lstlisting}
&
\begin{lstlisting}[language=diff]
public void play() {
-   if (mAudioManager.requestAudioFocus(
-           mAudioFocusListener,
-           AudioManager.STREAM_MUSIC,
-           AudioManager.AUDIOFOCUS_GAIN) !=
-       AudioManager.AUDIOFOCUS_REQUEST_GRANTED){
-       return;
-   }
+   int res;
+   int arg1=AudioManager.STREAM_MUSIC;
+   int arg2=AudioManager.AUDIOFOCUS_GAIN;
+   res = mAudioManager.requestAudioFocus (mAudioFocusListener, arg1, arg2);
+   if (res != AudioManager
+       .AUDIOFOCUS_REQUEST_GRANTED) {
+       return;
+   }
}
\end{lstlisting}
\\ \hline
3. Structurally \newline different: method argument
&
\begin{lstlisting}
protected String getInputDataString() {
    return String.format("\%02d:\%02d",
    timePicker.getCurrentHour(),
    timePicker.getCurrentMinute());
}
\end{lstlisting}
&
\begin{lstlisting}[language=diff]
protected String getInputDataString() {
+   int hour;
+   hour = timePicker.getCurrentHour();
+   return String.format("%02d:%02d", hour,
+   timePicker.getCurrentMInute());
-   return String.format("\%02d:\%02d",
-   timePicker.getCurrentHour(),
-   timePicker.getCurrentMinute());
}
\end{lstlisting}
\\ \hline
4. Structurally \newline different: arithmetic operand
&
\begin{lstlisting}
public void displayTime(View view) {
    String time = timePicker.getCurrentHour()
       + ":" + timePicker.getCurrentMinute();
    Toast.makeText(this, time,
        Toast.LENGTH_SHORT).show();
}
\end{lstlisting}
&
\begin{lstlisting}[language=diff]
public void displayTime(View view) {
+   int hour;
+   hour = timePicker.getCurrentHour();
+   String time = hour + ":" +
+     timePicker.getCurrentMinute();
-   String time = timePicker.getCurrentHour()
-      + ":" + timePicker.getCurrentMinute();
    Toast.makeText(this, time,
        Toast.LENGTH_SHORT).show();
}
\end{lstlisting}
\\ \hline
5. Direct \newline assignment \newline after declaration
&
\begin{lstlisting}
public Schedule generate() {
    TimePicker timePicker = (TimePicker) activity
        .findViewById(R.id.timePicker);
    int hours = timePicker.getCurrentHour();
    int minutes = timePicker.getCurrentMinute();
    SeekBar seekBar = (SeekBar) activity
        .findViewById(R.id.setLuminosity);
    int luminosity = seekBar.getProgress();
    return new Schedule(hours, minutes,
                        luminosity);
}
\end{lstlisting}
&
\begin{lstlisting}[language=diff]
public Schedule generate() {
    TimePicker timePicker = (TimePicker) activity
        .findViewById(R.id.timePicker);
+   int hours;
+   hours = timePicker.getCurrentHour();
-   int hours = timePicker.getCurrentHour();
    int minutes = timePicker.getCurrentMinute();
    SeekBar seekBar = (SeekBar) activity
        .findViewById(R.id.setLuminosity);
    int luminosity = seekBar.getProgress();
    return new Schedule(hours, minutes,
                        luminosity);
}
\end{lstlisting}
\\ \hline
6. Complex \newline expressions
&
\begin{lstlisting}
protected void onClick() {
    ...
    Dialog d = builder.create();
    d.show();
    ((TextView) d.findViewById(android.R.id.
      message))
        .setTextAppearance(getContext(),
        android.R.style.TextAppearance_Small);
    ...
}
\end{lstlisting}
&
\begin{lstlisting}[language=diff]
protected void onClick() {
    ...
    Dialog d = builder.create();
    d.show();
-   ((TextView) d.findViewById(android.R.id.
-     message))
-       .setTextAppearance(getContext(),
   -    android.R.style.TextAppearance_Small);
+   Context c = getContext();
+   int i = android.R.style.
+         TextAppearance_Small;
+   TextView t = ((TextView) d.findViewById
+     (android.R.id.message));
+   t.setTextAppearance(c, i);
    ...
}
\end{lstlisting}
\\ \hline


\end{tabular}
\end{table*}

\begin{table}
	\caption{Categories of failed updates that are unhandled by our mitigations}\label{tab:mitigatefail}
\centering
\begin{tabular}{|p{.08\textwidth}|p{.35\textwidth}|}
\hline
\textbf{Listing}
  &
  \textbf{Original Code}
 \\ \hline
7. Incomplete support: inheritance
&
\begin{lstlisting}
public class TestMediaPlayer extends MediaPlayer {
  public TestMediaPlayer() {
    setAudioStreamType(AudioManager.STREAM_MUSIC);
  }
  public TestMediaPlayer(Context testContext,
    int withResource) throws Exception {
    this();
    AssetFileDescriptor afd =
      testContext.getResources()
      .openRawResourceFd(withResource);
    assertNotNull(afd);
    setDataSource(afd.getFileDescriptor(),
                  afd.getStartOffset(),
                  afd.getLength());
    afd.close();
    prepare();
  }
}
\end{lstlisting}
\\ \hline
8. Incomplete support: static
&
\begin{lstlisting}
private static AudioManager mAudioManager;
private static OnAudioFocusChangeListener afChangeListener;
public static void requestAudioFocus
  (Context context) {
  if(mAudioManager == null){
    mAudioManager = (AudioManager)
      context.getApplicationContext()
      .getSystemService
      (Context.AUDIO_SERVICE);
  }
  mAudioManager.requestAudioFocus(
    afChangeListener,
    AudioManager.STREAM_MUSIC,
    AudioManager.AUDIOFOCUS_GAIN);
}
\end{lstlisting}
\\ \hline
9. Incomplete support: final
&
\begin{lstlisting}
private final Paint paint;
@Override protected void onDraw(Canvas canvas) {
  super.onDraw(canvas);
  canvas.drawColor(Color.GREEN);
  canvas.saveLayer(0, 0, canvas.getWidth(),
    canvas.getHeight(), paint,
    Canvas.ALL_SAVE_FLAG);
  canvas.restore();
}
\end{lstlisting}
\\ \hline
 \end{tabular}
 \end{table}

  \lstdefinelanguage{text}{
	basicstyle=\ttfamily\extrabold\scriptsize,
	identifierstyle=\color{black},
	aboveskip=5pt,
	belowskip= 5pt
}

\lstdefinelanguage{diff}{
	basicstyle=\ttfamily\extrabold\scriptsize,
	morecomment=[f][\color{diffstart}]{@},
	morecomment=[f][\color{diffincl}]{+},
	morecomment=[f][\color{diffrem}]{-},
	keepspaces=true,
	identifierstyle=\color{black},
	aboveskip=5pt,
	belowskip= 5pt
}
