The goal of our replication study is to determine whether the observed
effectiveness of AppEvolve generalizes to a wider range of apps.

\subsection{Dataset}
Our replication study focuses on the same set of deprecated methods as the
original evaluation of AppEvolve and the same training data, but considers
a larger set of apps that use these deprecated methods.  Specifically, we
include additional mobile apps that use the same deprecated APIs as in the
original AppEvolve dataset and that do not use the APIs that are suggested
to replace the deprecated APIs.  We find the new mobile apps by querying
GitHub Code Search\footnote{\url{https://github.com/search}} using the
names of the APIs. GitHub Code Search returns a list of ranked files
matching the query. Since Github Code Search only supports textual queries,
it returns many files that do not actually use the API methods that were
queried for (false positives).  Thus, we manually check the returned files
to confirm that they actually contain the usages of the desired deprecated
APIs and do not contain usages of their replacement APIs. Finally, we
randomly selected 54 API usages each from a different app in as our
dataset.  The usages are shown in Table~\ref{TODO}.

We note that although the AppEvolve training data also comes from GitHub,
there is no possible overlap with our test set.  GitHub Code Search only
indexes the latest version of each repository.  The training data consists
of apps where the latest version uses the replacement API, while our test
data consists of apps that do not use the replacement API, so no overlap is
possible.

\subsection{Procedure}
We need to create an AppEvolve configuration for each app in our
dataset. To do so, we carefully read and understand the existing AppEvolve
configurations that were used in the original AppEvolve experiments. We
also asked the first author of AppEvolve to confirm how to configure
AppEvolve for the various apps correctly. Finally, we ran our experiments
in the virtual machine environment provided by the AppEvolve
authors.\footnote{\url{https://sites.google.com/view/appevolve}}

After configuring AppEvolve for the apps, we ran AppEvolve on them. We
recorded the number of applicable and failed updates. For the failed
updates, we categorized them using card sorting\cite{...}. In card sorting,
we perform multiple passes on the failed update data. For the first pass,
we put each of the failed updates into a category created based on our
understanding of the reason for the failed update. For subsequent passes,
we reevaluated the categories. We might rename a category to be more
descriptive of the problem that occurs in the set of updates belonging to
the category or merge related categories into one. These steps were
repeated until there were no more changes to the categories.
