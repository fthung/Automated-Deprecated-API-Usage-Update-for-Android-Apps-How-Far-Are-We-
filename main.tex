\documentclass[conference]{IEEEtran}
\IEEEoverridecommandlockouts
% The preceding line is only needed to identify funding in the first footnote. If that is unneeded, please comment it out.
\usepackage{cite, balance}
\usepackage{amsmath,amssymb,amsfonts}
\usepackage{algorithmic}
\usepackage{algorithm2e}
\usepackage{graphicx}
\usepackage{textcomp}
\usepackage{xcolor}
\usepackage{url}
\usepackage{array}
\usepackage{multirow}
\usepackage{amssymb}
\def\BibTeX{{\rm B\kern-.05em{\sc i\kern-.025em b}\kern-.08em
    T\kern-.1667em\lower.7ex\hbox{E}\kern-.125emX}}

\definecolor{red}{HTML}{9B0000}
\definecolor{lightred}{HTML}{FF5131}
\definecolor{green}{HTML}{006400}
\definecolor{lightgreen}{HTML}{9CFF57}
\definecolor{purple}{HTML}{7200CA}
\definecolor{verylightgrey}{HTML}{F1F1F1}

\definecolor{diffstart}{named}{blue}%{Grey}
\definecolor{diffincl}{named}{green}
\definecolor{diffrem}{named}{red}

\newcommand{\extrabold}{\bfseries}

\usepackage{listings}
  \lstdefinelanguage{text}{
	basicstyle=\ttfamily\extrabold\scriptsize,
	identifierstyle=\color{black},
  }

\usepackage{listings}
  \lstdefinelanguage{diff}{
	basicstyle=\ttfamily\extrabold\scriptsize,
	morecomment=[f][\color{diffstart}]{@},
	morecomment=[f][\color{diffincl}]{+},
	morecomment=[f][\color{diffrem}]{-},
        keepspaces=true,
	identifierstyle=\color{black},
  }



% Very convenient to add comments on the paper. Just set the boolean
% to false before sending the paper:
\usepackage{ifthen}
\newboolean{showcomments}
\setboolean{showcomments}{true}
%\setboolean{showcomments}{false}
\ifthenelse{\boolean{showcomments}}
{ \newcommand{\mynote}[2]{\textcolor{red}{
    \fbox{\bfseries\sffamily\scriptsize#1}
    {\small$\blacktriangleright$\textsf{\emph{#2}}$\blacktriangleleft$}}}}
{ \newcommand{\mynote}[2]{}}

\newcommand{\jl}[1]{\mynote{Julia}{#1}}
\newcommand{\lx}[1]{\mynote{JLX}{#1}}
\newcommand{\ft}[1]{\mynote{Ferdian}{#1}}

\hyphenation{App-Evolve}

\begin{document}

\def \toolname {AppEvolve}
\title{Automated Deprecated-API Usage Update for Android Apps: How Far Are We?}

\author{
\IEEEauthorblockA{
Ferdian Thung\IEEEauthorrefmark{1},
Stefanus A. Haryono\IEEEauthorrefmark{1},
Lucas Serrano\IEEEauthorrefmark{2},
Gilles Muller\IEEEauthorrefmark{3},
Julia Lawall\IEEEauthorrefmark{3},
David Lo\IEEEauthorrefmark{1} and
Lingxiao Jiang\IEEEauthorrefmark{1}}

\IEEEauthorblockA{\IEEEauthorrefmark{1}School of Information Systems, Singapore Management University\\
Email: \{ferdianthung,stefanusah,davidlo,lxjiang\}@smu.edu.sg}
\IEEEauthorblockA{\IEEEauthorrefmark{2}Sorbonne University/Inria/LIP6\\
Email: lucas.serrano@lip6.fr}
\IEEEauthorblockA{\IEEEauthorrefmark{3}Inria\\
Email: \{Gilles.Muller,Julia.Lawall\}@inria.fr}
}

\maketitle

\begin{abstract}
As the Android API evolves, some API methods may be deprecated, to be
eventually removed.  App developers face the challenge of keeping their
apps up-to-date, to ensure that the apps work in both older and newer
Android versions.  Currently, AppEvolve is the state-of-the-art approach to
automate such updates, and it has been shown to be quite effective.  Still,
the number of experiments reported is moderate, involving only API usage
updates in 41 usage locations. In this work, we replicate the evaluation of
AppEvolve and assess whether its effectiveness is generalizable. Given the
set of APIs on which AppEvolve has been evaluated, we test AppEvolve on
other mobile apps that use the same APIs. Our experiments show that
AppEvolve fails to generate applicable updates for 81.48\% of our dataset,
even though the relevant knowledge for correct API updates is available in
the examples. We first categorize the limitations of AppEvolve that lead to
these failures.  We then propose a mitigation strategy that solves 81.82\%
of these failures by a simple refactoring of the app code to better
resemble the code in the examples. The refactoring usually involves
assigning the target API method invocation and the arguments of the target
API method into variables.  Indeed, we have also seen such transformations
in the dataset distributed with the AppEvolve replication package, as
compared to the original source code from which this dataset is derived.
Based on these findings, we propose some promising future directions.
\end{abstract}

\begin{IEEEkeywords}
API usage, program transformation, Android, mobile apps
\end{IEEEkeywords}

\section{Introduction}
Mobile apps are an integral part of a modern life. To keep up with increasing users' needs and demands, they may need to be frequently updated following the frequent updates of the underlying mobile operating system (OS)~\cite{bavota2014impact,han2012understanding,linares2013api,mcdonnell2013empirical,yang2018android}. App developers may need to maintain backward compatibility with different versions of the mobile OS to reach as many users as possible. To do so, developers need to update the APIs accordingly so that the same mobile apps can run without problems in the past and future versions of the mobile OS.

Unfortunately, updating APIs correctly is not always trivial. While API changes are described in documentation, examples of how to adapt to the API changes are not always available, especially if there is a need to maintain backward compatibility due to OS fragmentation~\cite{he2018understanding,li2018cid}. Moreover, the APIs to update may be used in many different locations in the mobile apps and thus updating them manually one by one is time consuming and error prone.

To automate the process of updating Android apps, Fazzini et al. proposed a technique named AppEvolve~\cite{fazzini2019automated}. To the best of our knowledge, AppEvolve is the state-of-the-art technique for automatic update of API-usage for Android apps. More specifically, it targets updates of deprecated-API usages. AppEvolve works by collecting information from existing code examples that already have gone through the update process. AppEvolve accepts as input a target Android app and information about API changes between two Android versions. It works in four phases. The first phase identifies code fragments in the target app that are affected by the API change. The second phase searches existing codebases for examples of the API change. The third phase generalizes each example into a form of generic code patch. These code patches are then ranked. In the fourth and final phase, it applies the generic code patches one at a time according to their ranking. Succesful application means that AppEvolve can produce applicable updates. These applicable updates are tested and labeled validated if the tests run without errors. The validated update is return to users. If none of the updates are validated, AppEvolve returns all applicable updates.

AppEvolve has been evaluated on a dataset involving 15 real-world apps and 20 real API changes. The dataset contains 41 usage locations of these real API changes. AppEvolve successfully produce applicable updates for 85\% of these API changes and 90.24\% of their usage locations. In this work, our goal is to analyze the characteristics of cases in which AppEvolve cannot generate applicable updates. We started by analyzing the original dataset that is made available by AppEvolve authors.\footnote{\url{https://sites.google.com/view/appevolve}} Next, we searched for additional mobile apps that use APIs in the original AppEvolve dataset. We applied AppEvolve to update these mobile apps and analyzed in detail the cases where it fails to update the apps.

Running AppEvolve on the combination of AppEvolve dataset and ours results in the generation of 10 applicable updates and 44 failed updates. Our analysis shows that AppEvolve produces applicable updates for a mobile app in cases where the API usage match the API change examples.  On the other hand, in most cases, AppEvolve fails to update a mobile app due to minor syntactical difference between invocation of the deprecated method in the example and the target app. It also fails when encountering edits that require modifications beyond method boundaries, unsupported programming language features, and no examples found in GitHub.

We mitigate cases of minor syntactical difference by performing simple refactoring in order to convert the target app code to resemble the example from which AppEvolve learns the edits from. After this simple refactoring, we found that AppEvolve can generate applicable updates for 81.82\% of cases where it previously failed to do so. Based on the discovered limitations in AppEvolve and our mitigations result, we propose promising future directions to improve updates of deprecated-API usage update in Android apps, such as normalizing code to a standard form coupled with identifier name recommendation.

The main contributions of this paper are as follows:
\begin{itemize}
    \item We have added additional Android apps to evaluate how well a state-of-the-art automatic API-usage update for Android apps tool named AppEvolve can actually perform an API-usage update in different situations.
    \item We categorized the limitations of AppEvolve and show instances of API usage instances in which these limitations can be demonstrated.
    \item We discuss how the limitations may possibly be overcome to improve the automatic API-usage update for Android apps.
    \item We release a replication package to allow easy replication and evaluation of our experiments and results.\footnote{\url{...}}
\end{itemize}

The remainder of this paper is structured as follows. Section~\ref{sec:approach} describes AppEvolve, the state-of-the-art work on automatic update of deprecated APIs in Android apps. Section~\ref{sec:replication} presents our replication settings. Section~\ref{sec:findings} presents our findings. Section~\ref{sec:mitigations} presents mitigations for some failed updates of Android app. In Section~\ref{sec:discuss}, we discuss our findings. Section~\ref{sec:related} presents some related work. Last but not least, Section~\ref{sec:conclusion} concludes and presents some future work.

\section{AppEvolve}\label{sec:approach}
We describe the problem statement, the approach, the dataset, and the main experimental results of AppEvolve.

\subsection{Problem Statement}
Consider two Android API versions: $oldAPIs = [m_1, m_2, ..., m_k]$ and $newAPIs = [m_1', m_2', ..., m_l']$. An API usage may involve one or more methods from either $oldAPIs$ or $newAPIs$. An API usage change from the $oldAPIs$ to the $newAPIs$ can be defined as a change from one or more methods from the $oldAPIs$ to one or more methods from the $newAPIs$, which can be written as $[m_1, m_2, ..., m_p] \rightarrow [m_1', m_2', ..., m_q']$. The goal of automatic update of Android apps is to update usages of $oldAPIs$ to $newAPIs$ based on the API usage change while also maintaining backward compatibility.

\begin{figure*}[htb]
	\centering
	\includegraphics[width=0.8\linewidth]{deprecated-api-example.png}
	\caption{An example of a deprecated API
	}
	\label{fig:deprecated_api_example}
\end{figure*}

To illustrate, consider the change in the Android API version 23 as shown in Figure~\ref{fig:deprecated_api_example}: the method \texttt{getCurrentHour()}, which returns the currently selected hour, in the range (0-23) was deprecated in version 23 of the Android API and the method \texttt{getHour()} was suggested as its replacement. The API usage change mapping is method \texttt{getCurrentHour()} to method \texttt{getHour()}.

Figure~\ref{fig:deprecated_api_update_example} shows an example of API usage changes that modify an usage of method \texttt{getCurrentHour()} to its replacement method \texttt{getHour()}. Given this example, one can learn relevant edits as shown in Figure~\ref{fig:deprecated_api_update_edits}. These edits update a deprecated method usage to a replacement method usage that is backward compatible. In these edits, backward compatibility is achieved by checking the version number of the mobile OS running the app. If the version is greater than or equal to a certain version number, the replacement method is used. Otherwise, the deprecated method is used.

\begin{figure}[htb]
\centering
\begin{lstlisting}[language=diff,numbers=none]
private long addEventToGCal() {
  ...
  int hourInt;
  int minInt;
- hourInt = timepicker.getCurrentHour();
- minInt = timepicker.getCurrentMinute();
+ if (android.os.Build.VERSION.SDK_INT >= 
+       android.os.Build.VERSION_CODES.M) {
+  hourInt = timepicker.getHour();
+  minInt = timepicker.getMinute();
+ } else {
+  hourInt = timepicker.getCurrentHour();
+  minInt = timepicker.getCurrentMinute();
+ }
  ...
}
\end{lstlisting}
\caption{An example of a deprecated API update}
\label{fig:deprecated_api_update_example}
\end{figure}

\begin{figure}[htb]
\centering
\begin{lstlisting}[language=diff,numbers=none]
- $T = $T.getCurrentHour();
+ if (android.os.Build.VERSION.SDK_INT >= 
+       android.os.Build.VERSION_CODES.M) {
+  $T = $T.getHour();
+ } else {
+  $T = $T.getCurrentHour();
+ }
\end{lstlisting}
\caption{Edits for updating deprecated method \texttt{getCurrentHour()}}
\label{fig:deprecated_api_update_edits}
\end{figure}

\subsection{Approach}
\toolname\ takes as input a target app to update and a mapping from a deprecated API method to its replacement API method(s). The framework of \toolname\ is shown in Figure~\ref{fig:framework}. The framework is divided into three phases: {\em API-Usage Analysis}, {\em Update Example Search}, and {\em API-Usage Update}. 

In the {\em API-Usage Analysis} phase, \toolname\ accepts as input an {\em API Usage Change} and a {\em Target App}. An {\em API Usage Change} describes the mapping from a deprecated API method to its corresponding replacement API method. A {\em Target App} is an app that contains the usage of the deprecated API method  and requires updates to make use of the replacement API method. Given these two inputs, \toolname\ pinpoints the location of a deprecated API usage in the {\em Target App}. In the {\em Update Example Search} phase, \toolname\ searches GitHub for examples of API usage updates that modify the usage of the deprecated API method to include the usage of the replacement API methods. In the {\em Update Example Analysis} phase, \toolname\ generates generic patches from the examples of API usage updates and ranks these patches. In the {\em API Usage Update} phase, \toolname\ applies the ranked patches and returns the {\em Evolved Target App} if the edits are successful.

\begin{figure*}[t]
	\centering
	\includegraphics[width=0.8\linewidth]{framework.png}
	\caption{Framework of AppEvolve}
	\label{fig:framework}
\end{figure*}

We next describe each of the above phase in detail.
\subsubsection{API-Usage Analysis}
\toolname\ finds the location where the deprecated API method specified in the {\em API Usage Change} is used inside the {\em Target App}. Consider the deprecated API method shown in Figure~\ref{fig:deprecated_api_example}, the {\em API Usage Change} is deprecated method \texttt{getCurrentHour()} is replaced with replacement method \texttt{getHour()}. Given this information, \toolname\ finds the location in the {\em Target App} that calls deprecated method \texttt{getCurrentHour()}. %Suppose that the {\em Target App}'s update location is as shown in Figure~\ref{fig:deprecated_api_update_target}, \toolname\ records this location for applying the updates later on.

\subsubsection{Update Example Search}
\toolname\ searches for apps in GitHub that use both the deprecated and the replacement API methods in its latest version. This is intended to find examples that produce a backward compatible Android app. We need both methods in an example since the Android app might be run in devices with either an older version of the Android OS (in which possibly only the deprecated method exists) or a newer version of the Android OS (in which possibly only the replacement method exists). Given these apps, \toolname\ finds changes that add the replacement API method to the app code that already contains the deprecated API method. These changes are the examples that can be used to update deprecated API method usages in other apps.

\subsubsection{Update Example Analysis}
Given examples found from GitHub, AppEvolve translates each example to a generic patch. It does so by identifying edits related to the API usage via intraprocedural forward and backward dependency analysis on the variables involved in the API usage. Variables that are used in statement affected by the edits but not defined by the edits themselves are considered as context variables. All variables in the edits are then abstracted. Given the generic patches, AppEvolve computes the the common core of these generic patches. The common core is defined as the longest subsequence of edits that are shared across the patches. The patches are then ranked based on their distance to the core.

\subsubsection{API-Usage Update}
Given a {\em Target App}, AppEvolve applies the generic patches according to their computed rank in the previous phase. When applying a patch, AppEvolve first finds mappings of context variables to variables in the {\em Target App}. For each such mapping, AppEvolve tries to apply the patch. If the edits are successfully applied, AppEvolve returns the {\em Evolved Target App}.

\subsection{Dataset}
For the target apps, AppEvolve paper used 15 real-world apps from the F-droid repository.\footnote{\url{https://fdroid.org}} The 15 real-world apps covers 20 API usages in 41 locations in total. AppEvolve selected five apps each for Android API versions 22, 23, and 25. For each API version, the API change is manually generated by reading the API
documentation. Using the API change, the API usage in each app is guaranteed to be: (1) different from the ones in the other apps and (2)
updated in the subsequent API version. 

\subsection{Results}
When applied to the 15 apps considered, AppEvolve was able to update 17 out of 20 API usages (85\% success rate) and 37 out of 41 of their occurrences across the apps.

\section{Replication Settings}\label{sec:replication}
The goal of our replication study is to determine whether the observed
effectiveness of AppEvolve generalizes to a wider range of apps.

\subsection{Dataset}
Our replication study focuses on the same set of deprecated methods as the
original evaluation of AppEvolve and the same training data, but considers
a larger set of apps that use these deprecated methods.  Specifically, we
include additional mobile apps that use the same deprecated APIs as in the
original AppEvolve dataset and that do not use the APIs that are suggested
to replace the deprecated APIs.  We find the new mobile apps by querying
GitHub Code Search\footnote{\url{https://github.com/search}} using the
names of the APIs. GitHub Code Search returns a list of ranked files
matching the query. Since Github Code Search only supports textual queries,
it returns many files that do not actually use the API methods that were
queried for (false positives).  Thus, we manually check the returned files
to confirm that they actually contain the usages of the desired deprecated
APIs and do not contain usages of their replacement APIs. Finally, we
randomly selected 54 API usages each from a different app in as our
dataset.  The usages are shown in Table~\ref{TODO}.

We note that although the AppEvolve training data also comes from GitHub,
there is no possible overlap with our test set.  GitHub Code Search only
indexes the latest version of each repository.  The training data consists
of apps where the latest version uses the replacement API, while our test
data consists of apps that do not use the replacement API, so no overlap is
possible.

\subsection{Procedure}
We need to create an AppEvolve configuration for each app in our
dataset. To do so, we carefully read and understand the existing AppEvolve
configurations that were used in the original AppEvolve experiments. We
also asked the first author of AppEvolve to confirm how to configure
AppEvolve for the various apps correctly. Finally, we ran our experiments
in the virtual machine environment provided by the AppEvolve
authors.\footnote{\url{https://sites.google.com/view/appevolve}}

After configuring AppEvolve for the apps, we ran AppEvolve on them. We
recorded the number of applicable and failed updates. For the failed
updates, we categorized them using card sorting\cite{...}. In card sorting,
we perform multiple passes on the failed update data. For the first pass,
we put each of the failed updates into a category created based on our
understanding of the reason for the failed update. For subsequent passes,
we reevaluated the categories. We might rename a category to be more
descriptive of the problem that occurs in the set of updates belonging to
the category or merge related categories into one. These steps were
repeated until there were no more changes to the categories.


\section{Findings}\label{sec:findings}
On our dataset, AppEvolve produces 10 applicable updates and 44 failed
updates. The original AppEvolve evaluation also showed 4 failed updates.
After investigating the 48 failed updates, we found some common reasons for
the failures, as described below. Table~\ref{table:data_statistic} shows
the number of occurrences of each of these issues.

% \begin{table}
% \caption{Update findings statistic}
% \begin{center}
% \begin{tabular}{ | p{8em} |c|c| } 
%  \hline
%  \textbf{Test Case Type} & \textbf{Detail of Cases} & \textbf{Number of Cases} \\ 
%  \hline
%  Work in AppEvolve & - & 10 \\ 
%  \hline
%  \multirow{7}{8em}{Fixed with simple refactor} & Total cases & 36 \\\cline{2-3} & Return statement & 6 \\\cline{2-3} & If statement & 1 \\\cline{2-3} & Method argument & 4 \\\cline{2-3} & Arithmetic operand & 3 \\\cline{2-3}  & Variable declaration & 12 \\\cline{2-3} & Complex expression & 20 \\\cline{2-3} 
%  \hline
%  \multirow{3}{8em}{Incomplete support of programming language features} & Inheritance & 2\\\cline{2-3} & Static modifier & 1 \\\cline{2-3} & Final modifier & 1\\\cline{2-3}
%  \hline
%  Unknown error & - & 4\\
%  \hline
% \end{tabular}
% \end{center}
% \label{table:data_statistic}
% \end{table}

\begin{table}
\caption{Update findings statistics}
\begin{center}
\begin{tabular}{ | p{12em} |c|c| } 
 \hline
 \textbf{Test Case Type} & \textbf{Subcategory} & \textbf{Occurrences} \\ 
 %\hline
 %Work in AppEvolve & - & 10 \\ 
 \hline
 \multirow{5}{12em}{Statements in the examples and at the target location are structurally different} & Return statement & 6 \\\cline{2-3} & If statement & 1 \\\cline{2-3} & Method argument & 4 \\\cline{2-3} & Arithmetic operand & 3 \\\cline{2-3}  & Declared variable & 12 \\\cline{2-3} 
 \hline
  \multirow{4}{12em}{Object and arguments of the deprecated API method are in the form of complex expressions} & - & 20\\&&\\&&\\&&\\
 \hline
 Edits beyond method boundaries & - & 3\\
\hline
 \multirow{3}{12em}{Incomplete support of programming-language features} & Inheritance & 2\\\cline{2-3} & Static modifier & 1 \\\cline{2-3} & Final modifier & 1\\\cline{2-3}
 \hline
 No examples & - & 1\\
 \hline
 Others & - & 4\\
 \hline
 \multicolumn{2}{|l|}{\bf Total} & 48\\
 \hline
\end{tabular}
\end{center}
\label{table:data_statistic}
\end{table}

\begin{enumerate}
\item {\em The statements in the examples and at the target location are
 structurally different.} AppEvolve infers edit operations at the statement
 level (insert of a statement, move of a statement, etc).  Accordingly,
 AppEvolve is not able to apply changes inferred about a method call found
 in one kind of statement to a method use found in another kind of
 statement.  In the training set used in the evaluation of AppEvolve, it
 often occurs that the invocation of the deprecated API method invocation
 is used as the right hand side of an assignment, while in our dataset the
 API method invocations occur in a variety of expression contexts.  We now
 present some examples:
\begin{itemize}
\item Return statement: 
In the code shown on the left side of Listing 1 in
Table~\ref{tab:mitigatesucc}, an invocation of the deprecated method {\tt
fromHtml} appears as part of a return statement of the {\tt getTitle}
method.

\item If statement test:
In the code shown on the left side of Listing 2 in
Table~\ref{tab:mitigatesucc}, an invocation of the deprecated method {\tt
requestAudioFocus} appears as a subexpression of the test expression of a
conditional.

\item Method argument:
In the code shown on the left side of Listing 3 in
Table~\ref{tab:mitigatesucc}, an invocation of the deprecated method {\tt
getCurrentHour} appears as the second argument of the method {\tt
String.format}.

\item Arithmetic operand:
In the code shown on the left side of Listing 4 in
Table~\ref{tab:mitigatesucc}, an invocation of the deprecated method {\tt
getCurrentHour} appears as the left argument of a concatenation with the
string {\tt ":"}.

\item Declared variable:
In the code shown on the left side of Listing 5 in
Table~\ref{tab:mitigatesucc}, an invocation of the deprecated method {\tt
getCurrentHour} appears as the initial value of a declared variable.  Even
though it involves an assignment, such code does not match examples where
the assignment involves a previously declared variable.
\end{itemize}

\item {\em Object and arguments of the deprecated API method are in the form of complex expressions.} The invocation of a deprecated method in the example that AppEvolve may learn from may use variables for the object and arguments of the method. These edits do not work when the object and arguments use complex expresssions. AppEvolve is not flexible enough to apply the edits correctly by treating these expressions as variables. The code in the left hand side of Listing 6 in Table~\ref{tab:mitigatesucc} shows an example of this case. The deprecated API method in this listing is {\tt setTextAppearance}.

\item {\em Edits beyond method boundaries.} AppEvolve cannot learn edits that modify program elements that reside outside the boundaries of the method containing the API usage to be updated. These edits include operations such as adding imports and adding fields to a class. Below is a snippet of an update example that involves edits beyond method boundaries.
\begin{lstlisting}[language=diff,numbers=none,caption=Edits that add imports and a private field,captionpos=b]
18a19
+ import android.location.GnssStatus;
22a24
+ import android.os.Build;
33a36,37
+ private GnssStatus.Callback callback;
41c45,51
- locationManager.addGpsStatusListener(listener);
+ if (android.os.Build.VERSION.SDK_INT >= 
            android.os.Build.VERSION_CODES.N) {
+   locationManager.registerGnssStatusCallback(callback);
+ }
+ else{
+   locationManager.addGpsStatusListener(listener);
+ }
\end{lstlisting}
The deprecated API in the above example is {\tt addGpsStatusListener}. Updating this API requires adding a private field named {\tt callback} of type {\em GnssStatus.Callback}. Unfortunately, since AppEvolve only learn edits inside a method containing the deprecated API usage, it misses this necessary addition.

\item {\em Incomplete support of programming language features.} AppEvolve fails to update cases in which the update involve programming language features such as:
\begin{itemize}
\item Inheritance: The code in the left hand side of Listing 7 in Table~\ref{tab:mitigatefail} shows an example of this case.  The deprecated API method is {\tt setAudioStreamType} of the class {\tt Media Player.} In the above snippet, {\tt TestMediaPlayer} extends {\tt MediaPlayer} and thus it inherits the {\tt setAudioStreamType} method. This method should be updated but AppEvolve does not seem to recognize it due to the use of inheritance.
\item Static modifier:  The code in the left hand side of Listing 8 in Table~\ref{tab:mitigatefail} shows an example of this case.  The deprecated API method {\tt requestAudioFocus} is invoked by {\tt mAudioManager}, which is a static field.

\item Final modifier: The code in the left hand side of Listing 9 in Table~\ref{tab:mitigatefail} shows an example of this case.  Variable {\tt paint} that is used as the deprecated API method {\tt saveLayer} fourth argument is a final field.

\end{itemize}

\item {\em No examples.} There is no example for such API update that can be found in GitHub. 

\item {\em Others.} These include cases that cannot be put to any category above.
\end{enumerate}


\newcolumntype{Y}{>{\centering\arraybackslash}X}

\lstset{
	language=text,numbers=none,
	breaklines=true,
	aboveskip=-7pt,
	belowskip= -6pt
}

\begin{table*}
\centering
\caption{Successful simple refactoring mitigations that allows AppEvolve to generate applicable updates}\label{tab:mitigatesucc}
\begin{tabular}{|p{.10\textwidth}|p{.40\textwidth}|p{.40\textwidth}|}
\hline
\textbf{Listing}
  &
  \textbf{Original Code}
  &
  \textbf{Refactored Code}
 \\ \hline
1. Structurally different: return statement
&
\begin{lstlisting}
Spanned getTitle(){
    return Html.fromHtml(title);
}
\end{lstlisting}
&
\begin{lstlisting}[language=diff]
Spanned getTitle(){
+   Spanned a;
+   a = Html.fromHtml(title);
-   return Html.fromHtml(title);
+   return a;
}
\end{lstlisting}
\\ \hline
2. Structurally different: if statement
&
\begin{lstlisting}
public void play() {
    if (mAudioManager.requestAudioFocus(
    mAudioFocusListener, AudioManager.STREAM_MUSIC,
    AudioManager.AUDIOFOCUS_GAIN) != 
    AudioManager.AUDIOFOCUS_REQUEST_GRANTED) {
        return;
    }
}
\end{lstlisting}
&
\begin{lstlisting}[language=diff]
public void play() {
-   if (mAudioManager.requestAudioFocus
-     (mAudioFocusListener, AudioManager.
-           STREAM_MUSIC,
-     AudioManager.AUDIOFOCUS_GAIN) !=
-     AudioManager.AUDIOFOCUS_REQUEST_GRANTED) {
-     return;
-   }
+   int res;
+   int arg1=AudioManager.STREAM_MUSIC;
+   int arg2=AudioManager.AUDIOFOCUS_GAIN;
+   res = mAudioManager.requestAudioFocus (mAudioFocusListener, arg1, arg2);
+   if (res != AudioManager
+     .AUDIOFOCUS_REQUEST_GRANTED) {
+     return;
+   }
}
\end{lstlisting}
\\ \hline
3. Structurally different: method argument
&
\begin{lstlisting}
protected String getInputDataString() {
    return String.format("\%02d:\%02d", 
    timePicker.getCurrentHour(), 
    timePicker.getCurrentMinute());
}
\end{lstlisting}
&
\begin{lstlisting}[language=diff]
protected String getInputDataString() {
+   int hour;
+   hour = timePicker.getCurrentHour();
+   return String.format("%02d:%02d", hour,
+   timePicker.getCurrentMInute());
-   return String.format("\%02d:\%02d", 
-   timePicker.getCurrentHour(), 
-   timePicker.getCurrentMinute());
}
\end{lstlisting}
\\ \hline
4. Structurally different: arithmetic operand
&
\begin{lstlisting}
public void displayTime(View view) {
    String time = timePicker.getCurrentHour() 
	    + ":" + timePicker.getCurrentMinute();
    Toast.makeText(this, time, 
        Toast.LENGTH_SHORT).show();
}
\end{lstlisting}
&
\begin{lstlisting}[language=diff]
public void displayTime(View view) {
+   int hour;
+   hour = timePicker.getCurrentHour();
+   String time = hour + ":" +
+     timePicker.getCurrentMinute();
-   String time = timePicker.getCurrentHour()
-      + ":" + timePicker.getCurrentMinute();
    Toast.makeText(this, time, 
        Toast.LENGTH_SHORT).show();
}
\end{lstlisting}
\\ \hline
5. Direct assignment after declaration
&
\begin{lstlisting}
public Schedule generate() {
    TimePicker timePicker = (TimePicker) activity
        .findViewById(R.id.timePicker);
    int hours = timePicker.getCurrentHour();
    int minutes = timePicker.getCurrentMinute();
    SeekBar seekBar = (SeekBar) activity
        .findViewById(R.id.setLuminosity);
    int luminosity = seekBar.getProgress();
    return new Schedule(hours, minutes, luminosity);
}
\end{lstlisting}
&
\begin{lstlisting}[language=diff]
public Schedule generate() {
    TimePicker timePicker = (TimePicker) activity
        .findViewById(R.id.timePicker);
+   int hours;
+   hours = timePicker.getCurrentHour();
-   int hours = timePicker.getCurrentHour();
    int minutes = timePicker.getCurrentMinute();
    SeekBar seekBar = (SeekBar) activity
        .findViewById(R.id.setLuminosity);
    int luminosity = seekBar.getProgress();
    return new Schedule(hours, minutes, luminosity);
}
\end{lstlisting}
\\ \hline
6. Complex expressions
&
\begin{lstlisting}
protected void onClick() {
    ...
    Dialog d = builder.create();
    d.show();
    ((TextView) d.findViewById(android.R.id.message))
        .setTextAppearance(getContext(), 
        android.R.style.TextAppearance_Small);
    ...
}
\end{lstlisting}
&
\begin{lstlisting}[language=diff]
protected void onClick() {  
    ...
    Dialog d = builder.create();
    d.show();
-   ((TextView) d.findViewById(android.R.id.
-     message))
-   .setTextAppearance(getContext(), 
-    android.R.style.TextAppearance_Small);
+   Context c = getContext();
+   int i = android.R.style.
+         TextAppearance_Small;
+   TextView t = ((TextView) d.findViewById
+     (android.R.id.message)); 
+   t.setTextAppearance(c, i);
    ...
}
\end{lstlisting}
\\ \hline


\end{tabular} 
\end{table*}

\begin{table}
	\caption{Failed categories that are unhandled by our mitigations}\label{tab:mitigatefail}
\centering
\begin{tabular}{|p{.08\textwidth}|p{.35\textwidth}|}
\hline
\textbf{Listing}
  &
  \textbf{Original Code}
 \\ \hline
7. Incomplete support: inheritance
&
\begin{lstlisting}
public class TestMediaPlayer extends MediaPlayer {
    public TestMediaPlayer() {
        setAudioStreamType(AudioManager.STREAM_MUSIC);
    }
    public TestMediaPlayer(Context testContext, 
        int withResource) throws Exception {
        this();
        AssetFileDescriptor afd = 
            testContext.getResources()
            .openRawResourceFd(withResource);
        assertNotNull(afd);
        setDataSource(afd.getFileDescriptor()
            , afd.getStartOffset(), afd.getLength());
        afd.close();
        prepare();
    }
}
\end{lstlisting}
\\ \hline
8. Incomplete support: static
&
\begin{lstlisting}
private static AudioManager mAudioManager;
private static OnAudioFocusChangeListener afChangeListener;
public static void requestAudioFocus
    (Context context) {
    if(mAudioManager == null){
        mAudioManager = (AudioManager) 
            context.getApplicationContext()
            .getSystemService
            (Context.AUDIO_SERVICE);
    }
    mAudioManager.requestAudioFocus(
        afChangeListener,
        AudioManager.STREAM_MUSIC,
        AudioManager.AUDIOFOCUS_GAIN);
}
\end{lstlisting}
\\ \hline
9. Incomplete support: final
&
\begin{lstlisting}
private final Paint paint;
@Override protected void onDraw(Canvas canvas) {
    super.onDraw(canvas);
    canvas.drawColor(Color.GREEN);
    canvas.saveLayer(0, 0, canvas.getWidth(),
        canvas.getHeight(), paint, 
        Canvas.ALL_SAVE_FLAG);
    canvas.restore();
}
\end{lstlisting}
\\ \hline
 \end{tabular}
 \end{table}

  \lstdefinelanguage{text}{
	basicstyle=\ttfamily\extrabold\scriptsize,
	identifierstyle=\color{black},
	aboveskip=5pt,
	belowskip= 5pt
}


\section{Mitigations}\label{sec:mitigations}
We tried to modify the target location in the app's source code so that
AppEvolve can produce an applicable update. To simplify the usages of the
deprecated apps, we refactored these usages such that each deprecated API
method invocation appears as the right hand side of an assignment
statement, and such that each argument of the deprecated method is a simple
variable, reminiscent of the three address code used in compiler
intermediate repreentations \cite{compiler}.  The right side of
Table~\ref{tab:mitigatesucc} shows the result of applying these mitigations
on the cases presented in Section~\ref{sec:findings} where running
AppEvolve on the transformed code produces applicable updates.

\begin{enumerate}
\item We refactor the code on the left hand of Listing 1 in Table~\ref{tab:mitigatesucc} by following the diff shown on the right hand side. In this refactored code, rather than directly returning the result of {\tt fromHtml} deprecated method invocation, we first assign it to a variable named {\tt a} of type {\tt Spanned}. %This allows AppEvolve to generate an applicable update to this piece of code.

\item We refactor the code on the left hand of Listing 2 in Table~\ref{tab:mitigatesucc} by following the diff shown on the right hand side. In this refactored code, rather than directly inserting {\tt AudioManager.STREAM\_MUSIC} and {\tt AudioManager.AUDIOFOCUS\_GAIN} constants as the second and third arguments of {\tt requestAudioFocus} deprecated method invocation, they are first assigned to variables named {\tt arg1} of type {\tt int} and {\tt arg2} of type {\tt int}, respectively. Moreover, the result of {\tt requestAudioFocus} deprecated method invocation is assigned to a variable  named {\tt res} of type {\tt AudioManager}, which is then used in an if condition.

\item We refactor the code in the left hand of Listing 3 in  Table~\ref{tab:mitigatesucc} by following the diff shown on the right hand side. In this refactored code, rather than directly inserting the result of invoking {\tt getCurrentHour} deprecated method to {\tt String.format} second argument, the result is first assigned to a variable named {\tt hour} of type {\tt int}.

\item We refactor the code on the left hand of Listing 4 in  Table~\ref{tab:mitigatesucc} by following the diff shown on the right hand side. In this refactored code, rather than concatenating the result of invoking {\tt getCurrentHour} deprecated method directly with ``:'' string, the result is first assigned to a variable named {\tt hour} of type {\tt int}.

\item We refactor the code on the left hand of Listing 5 in  Table~\ref{tab:mitigatesucc} by following the diff shown on the right hand side. In this refactored code, rather than directly assigning the result of invoking {\tt getCurrentHour} to a variable named {\tt hours} when it is declared, we declare the variable {\tt hours} first and then assign the result of invoking {\tt getCurrentHour} to the variable {\tt hours}.

\item We refactor the code on the left hand of Listing 6 in Table~\ref{tab:mitigatesucc} by following the diff shown on the right hand side. In this refactored code, rather than invoking {\tt setTextAppearance} deprecated method directly from the object returned by invoking {\tt findViewById} method, the returned object is first assigned to a variable named {\tt t} of type {\tt TextView}. {\tt setTextAppearance} method is then invoked from the variable {\tt t}.

\end{enumerate}

In essence, our mitigations refactor the deprecated method invocation in the target app to resemble the examples from which AppEvolve learns the edits to be applied to the target app. Our mitigations result in 36 successful updates out of the 44 failed updates (81.82\% success rate).


\section{Discussion}\label{sec:discuss}
In this section, we discuss some threats to validity, cases that are not
handled by our mitigations, and future directions for improving the
approach to updating deprecated API usage in Android apps.

\subsection{Threats to Validity}

One threat is related to whether we have configured AppEvolve correctly. To mitigate this threat, we have tried our best to run AppEvolve following the instructions given by the authors in the AppEvolve documentation. We have also asked AppEvolve's first author on how to configure AppEvolve correctly for new mobile apps. We can replicate all of the results on the original dataset, even when we try to redo the configurations from scratch.  We have also rechecked our configurations several times. Also, we have released a replication package\footnote{\url{https://sites.google.com/smu.edu.sg/appevolve-replication}} for others to check and validate.

Another threat is related to generalizability of the findings. We have added 54 new applications to evaluate the effectiveness of
AppEvolve in generating applicable updates. This translates to 54 API usage
locations on top of the 41 API usage locations in AppEvolve dataset, thus
more than doubling the number of API usage locations and more than triple the number of apps as compared to the dataset used in the original work. We
believe this is sufficient to understand the capabilities of AppEvolve, as
44 of the 54 API usages that we have added uncover limitations of
AppEvolve. There might be other cases that AppEvolve cannot handle, but we
believe that we have found many of them.


\subsection{Cases Unhandled by Our Mitigation Strategy}

There are 8 cases in which our mitigation strategy was not successful. For two cases where our mitigations do not work, we manage to manually alter the target code such that
AppEvolve can successfully apply the learned edits. We show these two cases below.

For one case, we modified the code on the left hand side of Listing 7 in Table~\ref{tab:mitigatefail} to an equivalent one following the diff shown below. In the modified code, a static field {\tt afChangeListener} of type {\tt AudioManager.OnAudioFocusChangeListener} is directly put as the first argument of {\tt requestAudioFocus} and AppEvolve cannot generate an applicable update. When {\tt afChangeListener} is assigned to a local variable named {\tt af}, AppEvolve can generate an applicable update.

\vspace{0.5cm}\begin{lstlisting}[language=diff,numbers=none]
private static AudioManager mAudioManager;
private static OnAudioFocusChangeListener afChangeListener;
public static void requestAudioFocus
    (Context context) {
    if(mAudioManager == null){
        mAudioManager = (AudioManager) context
            .getApplicationContext()
            .getSystemService(Context.AUDIO_SERVICE);
    }
-   mAudioManager.requestAudioFocus(
-       afChangeListener,
-       AudioManager.STREAM_MUSIC,
-       AudioManager.AUDIOFOCUS_GAI
+   int res;
+   int a = AudioManager.STREAM_MUSIC;
+   int b = AudioManager.AUDIOFOCUS_GAIN;
+   AudioManager am = mAudioManager;
+   AudioManager.OnAudioFocusChangeListener
+        af = afChangeListener;
+    res = am.requestAudioFocus(af,a,b);
}
\end{lstlisting}


\vspace{0.5cm} For another case, we modify the code on the left hand of Listing 8 in Table~\ref{tab:mitigatefail} to an equivalent one following the diff shown below.

\vspace{0.5cm}\begin{lstlisting}[language=diff,numbers=none]
private Paint paint;
@Override protected void onDraw(Canvas canvas) {
    super.onDraw(canvas);
    canvas.drawColor(Color.GREEN);
+   float a = 0;
+   float b = 0;
+   float c = getWidth();
+   float d = getHeight();
+   int flag = Canvas.CLIP_SAVE_FLAG;
+   canvas.saveLayer(a,b,c,d, paint, flag);
-   canvas.saveLayer(0, 0, canvas.getWidth(),
-       canvas.getHeight(), paint,
-       Canvas.ALL_SAVE_FLAG);
    canvas.restore();
  }
\end{lstlisting}

\vspace{0.5cm}In the above listing, a field named {\tt paint} of type {\tt Paint} has a {\tt final} modifier. Removing this modifier results in an applicable update. It suggests that AppEvolve does not support {\tt final} modifier.

\subsection{Future Directions}

Our mitigations can fix 81.82\% of the failed updates that we found,
showing that even simple modifications of the target apps can allow
edits that are learned from examples of API usage to be applied
successfully. This technique can possibly be generalized by normalizing the
code into a standard form. Edits can then be learned from this standard
form. When the edit is applied to a new piece of code, that target code
should also be normalized, which should minimize the variations in the code
due to simple refactorings. If the edits are successfully applied, the
resulting code can then be refactored again to restore the coding
style of the developers.  Our mitigations represent a subset of the normalizations that are
possible.

Related to the last point on refactoring resulting code to follow the coding style of developers, we perform a simple experiment to compare the output of AppEvolve after our simple refactoring with code that are manually updated to deal with a set of deprecated API.  We
asked a software engineer who is not a co-author of this paper to update the code shown on the left hand side of Listing ... in Table~\ref{tab:mitigatesucc}. The code generated by AppEvolve and by the engineer are shown below.

%Buse and Weimer~\cite{Buse:2008:MSR:1390630.1390647} propose a metric of
%software readability that are based on a set of simple local features. They
%also provide an automatic readability measure model that is built on 12,000
%code readability judgments. We run this model and compare the readability
%measure of the API update created by AppEvolve against that of the update
%created by the human.  Scores range from 0 to 1 where, 0 is least and 1 is
%most readable.

\vspace{0.2cm}\noindent {\bf AppEvolve's Update}
\vspace{0.5cm}\begin{lstlisting}[language=diff,numbers=none]
public void displayTime(View view){
  int varInt1;
  int varInt2;
  if (android.os.Build.VERSION.SDK_INT >= android.os.Build.VERSION_CODES.M) {
    varInt1=timePicker.getHour();
  }
 else {
    varInt1=timePicker.getCurrentHour();
  }
  if (android.os.Build.VERSION.SDK_INT >= android.os.Build.VERSION_CODES.M) {
    varInt2=timePicker.getMinute();
  }
 else {
    varInt2=timePicker.getCurrentMinute();
  }
  String time=varInt1 + ":" + varInt2;
  Toast.makeText(this,time,Toast.LENGTH_SHORT).show();
}
\end{lstlisting}


\vspace{0.7cm}\noindent {\bf Engineer's Update}
\vspace{0.5cm}\begin{lstlisting}[language=diff,numbers=none]
public void displayTime(View view) {
    String time;
    if (android.os.Build.VERSION.SDK_INT >= 23) {
        // only for gingerbread and newer versions
        time = timePicker.getHour() + ":" + timePicker.getMinute();
    } else {
        time = timePicker.getCurrentHour() + ":" + timePicker.getCurrentMinute();
    }
    Toast.makeText(this, time, Toast.LENGTH_SHORT).show();
}
\end{lstlisting}

%\jl{the following two paragraphs have to be merged.}

\vspace{0.5cm}The code made manually by the engineer is different from the code produced
by AppEvolve. Indeed, developers may prefer the code made the
engineer. Future work may want investigate methods that can effectively
refactor AppEvolve generated code to better fit into the coding style of
developers. One step would be to choose good names for the newly introduced
variables. A number of recent studies use approaches based on deep learning
and mining software repositories to recommend method and class
names~\cite{allamanis2015suggesting}. These approaches can potentially be extended for inferring good
variable names in our context.

%As you can see, the code made by the engineer is different than AppEvolve's. Developers may prefer the code made by the engineer. Future work may want to investigate methods that can effectively refactor AppEvolve generated code to better fit into the coding style of developers. One future work that can be done is to design methods to infer good names to newly introduced variables. There have been a number of studies that use deep learning and mining software repository approaches to recommend method and class names~\cite{allamanis2015suggesting}. These work can potentially be extended for inferring good variable names.


%\jl{why was this paragraph removed?}
Moreover, in this work, we have assumed it is safe to move a method
invocation from a location to another earlier location. Such a rewriting
may have unintended consequences in the presence of side effects, if the
transformation changes the order in which code is executed.  For example,
suppose we move a method invocation $M_3$ that appears as a third argument
of another method invocation $M$. For this case, if the evaluation of the
first and the second arguments of $M$ influence the internal state of the
program that affects the outcome of $M_3$, the refactoring that we did may
lead to an erroneous program state. In the future, we may want to extend
the line of work on side effect analysis and purity inference, e.g., to
deal with this issue. Note that, in our dataset, we do not have such a
case.  Indeed, it is typically not considered to be good programming style
to rely on the order of evaluation of function arguments.





%The difference in readability is very small. Indeed, the human developer
%also implicitly performed the normalization, and the differences between
%the variants are due to the choice of variable names, the spacing around
%function arguments and the degree of indentation.  Recent research has
%proposed strategies to automate the selection of new variable
%names~\cite{...}.  The other issues can be addressed by tools such as
%\jl{indent?  Apparently it only targets C programs} that are
%parameterizable according to the desired coding style.


\section{Related Work}\label{sec:related}
%In this section, we present prior works on API deprecation, API evolution, and API migration.
\subsection{API Deprecation}
There are a lot of works that focus on studying API deprecation~\cite{zhou2016api,kapur2010refactoring,raemaekers2014semantic,brito2016developers,robbes2012developers,sawant2016reaction,sawant2018features,li2018characterising}. Zhou and Walker~\cite{zhou2016api} found that, in practice, deprecating APIs does not always follow {\em deprecated-replace-remove} cycle, such as many deprecated APIs are undeprecated. They also developed a tool to warn about deprecated API usages in StackOverflow posts. Kapur et al.~\cite{kapur2010refactoring} found that APIs might be removed without being marked as deprecated. Raemaekers et al.~\cite{raemaekers2014semantic} discovered that some Java artifacts on the Maven Central Repository never remove deprecated APIs. Brito et al.~\cite{brito2016developers} showed that not all APIs are annotated with replacement messages. Robbes et al.~\cite{robbes2012developers} analyzed the Smalltalk ecosystem and showed that some API changes caused by deprecation can substantially impact the ecosystem. This study was replicated on the Java ecosystem and similar results were reported~\cite{sawant2016reaction,sawant2018reaction}, except that the number of deprecated API replacements was higher in the Smalltalk ecosystem. Sawant et al.~\cite{sawant2018features} created a taxonomy containing 12 reasons for deprecation and developed an approach to automatically classify them. Li et al.~\cite{li2018characterising} performed an exploratory study on characterizing Android APIs. They found that, among other things, deprecated Android APIs are not always consistently annotated and documented, and they are also regularly cleaned up. They have also developed a prototype tool that can generate API replacement mappings from the Android framework source code. %However, the automatically generated mappings may not be correct. In contrast, we have manually curated our dataset of 647 deprecated method mappings.
%\lx{"these mappings"? It sounds like the mappings auto-generated by Li et al. are correct? May be better to give some numbers to show how our curated dataset is different from their mappings.}

All of the above studies aim to understand API deprecation. In this work,
we aim to understand the applicability of a state-of-the-art approach for
automatic update of Android apps, which updates uses of deprecated methods to
corresponding uses of their replacement methods.

%% \subsection{API Evolution}

%% There are many works that focus on supporting API
%% evolution~\cite{henkel2005catchup,schafer2008mining,wu2010aura,dagenais2009semdiff,dagenais2011recommending,meng2012history,yu2017api}. Henkel
%% et al.~\cite{henkel2005catchup} created a tool named CatchUp! that records
%% developers' refactoring actions when evolving uses of an API within an
%% Integrated Development Editor (IDE) and can replay them afterwards. Xing et
%% al.~\cite{xing2007api} proposed Diff-CatchUp!, a tool that recommends
%% plausible replacements for obsolete API usages following example changes in
%% the API change history.  Sch{\"a}fer et al.~\cite{schafer2008mining}
%% proposed to mine API replacement rules for API evolution by analyzing
%% client code that has performed the replacements. Dagenais and
%% Robillard~\cite{dagenais2009semdiff,dagenais2011recommending} proposed an
%% approach named SemDiff that learns how an API adapts to its own changes and
%% uses this information to recommend a set of method replacements to client
%% programs. Wu et al.~\cite{wu2010aura} developed a tool named AURA to mine
%% API replacement rules using a combination of method call dependency
%% analysis and text similarity. Meng et al.~\cite{meng2012history} proposed
%% HiMA, a tool that infers API replacement rules between two versions of a
%% framework by aggregating API replacement rules for each pair of consecutive
%% revisions in the framework history between the two versions. Yu et
%% al.~\cite{yu2017api} develop an approach named AUC-Miner, which mines API
%% replacement rules by employing context information to refine method call
%% dependency analysis.

%% %In this work, given the API replacement rules, our approach generates transformation rules, in the form of semantic patches, that can be used to perform the API replacement. Thus, our approach complements approaches that mine API replacement rules.

%% \subsection{API Migration}

%% Many works have been proposed to support API migration in varying
%% situations~\cite{hora2015apiwave,zhou2016api,nguyen2010graph,nguyen2014statistical,nita2010using,lamothe2018a4}. Nita
%% and Notkin~\cite{nita2010using} proposed the use of twinning (i.e., a
%% technique to specify program changes without modifying the program
%% directly) to migrate a program to use alternative APIs. Hora and
%% Valente~\cite{hora2015apiwave} proposed Apiwave, a tool for tracking API
%% popularity and migration. Zhong et al.~\cite{zhou2016api} developed a tool
%% called MAM, which mines API mappings for language migration from programs
%% with two v<ersions in two programming languages. Nguyen et
%% al.~\cite{nguyen2010graph} developed LIBSYNC, which employs a graph based
%% approach to migrate API usage in client code to the new version of the API
%% by learning from clients that have performed the adaptations. Nguyen et
%% al.~\cite{nguyen2014statistical} proposed StaMiner, a statistical model
%% based approach to mine API mappings for migrating programs from one
%% language to another. Lamonthe et al.~\cite{lamothe2018a4} proposed an
%% approach that learns API migration patterns from code examples and can
%% apply these patterns to migrate deprecated APIs.

\subsection{Replication Studies}

There have been a number of replication studies in the software engineering
domain \cite{Chen:2017:CLP:3042021.3042046,
  Greiler:2015:COS:2820518.2820522, Akbarinasaji:2018:PBT:3174380.3174639,
  Dinh-Trong:2005:FPR:1079843.1080069, howdopython} that have not always
confirmed the original results. Chen and
Jiang~\cite{Chen:2017:CLP:3042021.3042046} replicated a study by Yuan et
al.~\cite{Yuan:2012:CLP:2337223.2337236} of logging practices.  In contrast
to the observations of Yuan et al., Chen and Jiang found that bug reports
without a log message take a shorter time to resolve than bug reports that
include a log.  Greiler et al.~\cite{Greiler:2015:COS:2820518.2820522}
replicated the work of Bird et al.~\cite{Bird:2011:DTM:2025113.2025119} on
the correlation between code ownership and software quality.  Greiler et
al. used new and refined code ownership metrics and prediction
models. Akbarinasaji et al.~\cite{Akbarinasaji:2018:PBT:3174380.3174639}
replicated and reinforced the finding on the bug fixing time estimation
model by showing similar result with the previous work. A replication study
on open source development by Trong et
al.~\cite{Dinh-Trong:2005:FPR:1079843.1080069} found new findings from the
previous work by Mockus et al.~\cite{Mockus:2002:TCS:567793.567795}. They
supported some of the previous hypotheses and proposed revisions on hypothesis related to the need of formal arrangement for work coordination and on hypothesis regarding the number of core developers on open source project. A
replication study by Orru et al.~\cite{howdopython} conducted an analysis
of the use of inheritance in Python systems that was previously done on
Java. Their result shows that compared on the previous findings on Java,
Python has more classes that are inherited from but fewer classes that inherit from other
classes.

%Among the API migration work above, Lamonthe et al.'s work is the closest to our work. Our work also deals with API migration, particularly replacing deprecated APIs. However, different than their work, our work focuses on the scenario in which there are no examples. 

% \subsection{API Usage Recommendation}
% Several approaches have been developed to recommend and explore how an API can be used~\cite{glassman2018visualizing}. Glassman et al.~\cite{glassman2018visualizing} proposed EXAMPLORE, an tool that summarizes hundreds of code examples for an API in a single view.


\subsection{Program Transformation}

Normalization of code to improve the results of program transformations has
a long history.  It was extensively used in the early 1990s in the context
of partial evaluation (specialization of programs to the values of some
known inputs), in the form of {\em binding-time improvements}, in order to
bring known values closer together, to allow code simplifications
\cite{some,things}.  \jl{X} and \jl{Y} have more generally studied and
formalized the impact of program transformations on program optimizations
\cite{paper_from_david}.

AppEvolve is also part of a line of research that has recently been
attracting increasing interest on inferring program transformations from
one or more examples.  LASE~\cite{Meng:2013:LLA:2486788.2486855} creates an
edit script from examples, locate the edit locations \jl{really?}, and do
the code transformation
automatically. REFAZER~\cite{Rolim:2017:LSP:3097368.3097417} is a technique
for automatic program transformation generation that is build based on code
edits performed by developers as input-output
examples. Genesis~\cite{Long:2017:AIC:3106237.3106253} automatically infers
patch generation transforms from previous successful
patches. PHOENIX~\cite{Bavishi:2019:PAD:3338906.3338952} is a fully
automated pipeline system that mines and cleans patches from examples and
learns generalized executable repair strategies by leveraging a novel
Domain Specific Language
(DSL). Spdiff~\cite{Andersen:2012:SPI:2351676.2351753} infers
transformation specifications from examples of both original and updated
code.
\jl{These descriptions should be improved.  We don't see how the
  capabilities of these tools relate to each other, not how they relate to
  the capabilities of AppEvolve.}

\subsection{Program Transformation}

\jl{alternate section, need to merge}

Research on program transformation have been done in multiple works~\cite{Visser:2001:SLP:647200.718711, Meng:2013:LLA:2486788.2486855, Rolim:2017:LSP:3097368.3097417,Long:2017:AIC:3106237.3106253, Bavishi:2019:PAD:3338906.3338952, Andersen:2012:SPI:2351676.2351753}. Visser~\cite{Visser:2001:SLP:647200.718711, 10.1007/978-3-540-87875-9_13, Lee:2013:DRI:2486788.2486792, Nguyen:2019:GMI:3339505.3339608} proposed Stratego, a language for program transformation that is based on rewriting strategies. Stratego aims to make reusable transformation rules that can be used in multiple transformations by separating the strategies from the transformation rules. Meng et al.~\cite{Meng:2013:LLA:2486788.2486855} developed LASE, a tool that can create edit script from examples, locate the edit locations, and do the code transformation automatically. Rolim et al.~\cite{Rolim:2017:LSP:3097368.3097417} proposed REFAZER, a technique for automatic program transformation generation which is build based on code edits performed by developers as input-output examples. Long et al.~\cite{Long:2017:AIC:3106237.3106253} proposed Genesis which can automatically infer patch generation transforms from previous successful patches. Bavishi et al.~\cite{Bavishi:2019:PAD:3338906.3338952} developed PHOENIX, a fully automated pipeline system that mines and clean patches from examples and learns generalized executable repair strategies by leveraging a novel Domain Specific Language (DSL). Andersen et al.~\cite{Andersen:2012:SPI:2351676.2351753} proposed the tool spdiff that can infer transformation specifications from examples of both original and updated code. Robbes and Lanza~\cite{10.1007/978-3-540-87875-9_13} presented a system that support example-based program transformation. The system takes an example manual change by developer and generalizes it to other application contexts. Lee et al.~\cite{Lee:2013:DRI:2486788.2486792} implemented DNDRefactoring, a tool that streamline the configuration and invocation of refactoring system through the use of direct manipulation via drag-and-drop. Nguyen et al.~\cite{Nguyen:2019:GMI:3339505.3339608} proposed a graph-based mining approach on detecting repetitive code changes called as CPatMiner. This approach aims to detect semantic code change patterns from large corpus and present them as graph.


\section{Conclusion and Future Work}\label{sec:conclusion}
AppEvolve is the state-of-the-art approach for automatic update of
deprecated-API usage in Android apps. Experiments previously reported for
AppEvolve have shown that it can generate applicable updates for
85.00\% of these API
changes and 90.24\% of their usage locations on mobile apps in the
AppEvolve dataset.

In this work, we evaluate whether this observed effectiveness is
generalizable. We add 54 additional mobile apps that use the APIs contained
in the AppEvolve dataset. Running AppEvolve on these mobile apps shows that
AppEvolve fails to generate applicable updates for 81.48\% of the mobile
apps. By analyzing these failed cases, we found that they failed mainly
due to:
\begin{enumerate}
    \item Statements in the examples and at the target location are structurally different.
    \item Object and arguments of the deprecated API method are in the form of complex expressions.
    \item Edits are required beyond method boundaries.
    \item Incomplete support of programming language features.
    \item No examples found in GitHub.
\end{enumerate}
We mitigate the first and the second categories by performing a simple
refactoring that modifies the code containing API usage in the target app
to resemble the one in the example. Our mitigations enable AppEvolve
to generate applicable updates for 81.82\% of the failed cases.

Our results suggest that the results previously obtained with AppEvolve do
not generalize beyond their dataset.  By studying and example in the
dataset and one of its corresponing change examples, it appears that the
developers of AppEvolve normalized the evaluated change examples and target
apps in a manner analogous to our mitigation, although this normalization
does not appear to be reported in the AppEvolve paper.  We propose that
applying such a mitigation systematically, combined with identifier name
recommendation and refactoring of the generated code to restore the
original coding style may be worth exploring for the future use of
AppEvolve.



\balance

\bibliography{references}
\bibliographystyle{plain}

\end{document}
