In this section, we discuss some threats to validity, the relation of our
results to those originally presented for AppEvolve, cases that are not
handled by our mitigations, and future directions for improving the
approach to updating deprecated API usage in Android apps.

\subsection{Threats to Validity}

One threat is related to whether we have configured AppEvolve correctly. To
mitigate this threat, we have tried our best to run AppEvolve following the
instructions given by the authors in the AppEvolve documentation. We have
also asked AppEvolve's first author how to configure AppEvolve correctly
for new mobile apps. Based on the information obtained, we independently
reconstructed the configurations for the target apps in the original
dataset, and using these configurations were able to reproduce the results
reported for that dataset.  We have also rechecked our configurations for
the new dataset several times. We have released a replication package for
others to check and
validate.\footnote{\url{https://sites.google.com/smu.edu.sg/appevolve-replication}}

Another threat is related to the generalizability of the findings. We have added 54 new applications to evaluate the effectiveness of
AppEvolve in generating applicable updates. This translates to 54 API usage
locations on top of the 41 API usage locations in AppEvolve dataset, thus
more than doubling the number of API usage locations and more than tripling the number of apps as compared to the dataset used in the original work. We
believe this is sufficient to understand the capabilities of AppEvolve, as
44 of the 54 API usages that we have added uncover limitations of
AppEvolve. There might be other cases that AppEvolve cannot handle, but we
believe that we have found many of them.

\subsection{Cases Not Handled by Our Mitigation Strategy}

There are 8 failed updates in our dataset where our mitigation
strategy is not successful. For the cases of {\tt static} and {\tt final}
annotations, representing two failed updates where our mitigations do not
work, we manage to manually alter the target code such that AppEvolve can
successfully apply the learned edits. We present these cases below.

We modified the code on the left side of Listing 8 in
Table~\ref{tab:mitigatefail} following the diff shown below. In the
original code, a static field {\tt afChangeListener} of type {\tt
  Audio\-Manager.OnAudioFocusChangeListener} is directly put as the first
argument of {\tt requestAudioFocus} and AppEvolve cannot generate an
applicable update. When {\tt afChangeListener} is assigned to a local
variable named {\tt af} in addition to our mitigation, AppEvolve can
generate an applicable update.

%\vspace{\baselineskip}

\begin{lstlisting}[language=diff,numbers=none]
private static AudioManager mAudioManager;
private static OnAudioFocusChangeListener afChangeListener;
public static void requestAudioFocus
    (Context context) {
    if(mAudioManager == null){
        mAudioManager = (AudioManager) context
            .getApplicationContext()
            .getSystemService(Context.AUDIO_SERVICE);
    }
-   mAudioManager.requestAudioFocus(
-       afChangeListener,
-       AudioManager.STREAM_MUSIC,
-       AudioManager.AUDIOFOCUS_GAI
+   int res;
+   int a = AudioManager.STREAM_MUSIC;
+   int b = AudioManager.AUDIOFOCUS_GAIN;
+   AudioManager am = mAudioManager;
+   AudioManager.OnAudioFocusChangeListener
+        af = afChangeListener;
+    res = am.requestAudioFocus(af,a,b);
}
\end{lstlisting}

\vspace{\baselineskip}

We modified the code on the left side of Listing 9 in
Table~\ref{tab:mitigatefail} following the diff shown below.

%\vspace{\baselineskip}

\begin{lstlisting}[language=diff,numbers=none]
-private final Paint paint;
+private Paint paint;
@Override protected void onDraw(Canvas canvas) {
    super.onDraw(canvas);
    canvas.drawColor(Color.GREEN);
+   float a = 0;
+   float b = 0;
+   float c = getWidth();
+   float d = getHeight();
+   int flag = Canvas.CLIP_SAVE_FLAG;
+   canvas.saveLayer(a,b,c,d, paint, flag);
-   canvas.saveLayer(0, 0, canvas.getWidth(),
-       canvas.getHeight(), paint,
-       Canvas.ALL_SAVE_FLAG);
    canvas.restore();
  }
\end{lstlisting}

\vspace{\baselineskip}

In the original code, the field named {\tt paint} of type {\tt Paint} has a
{\tt final} modifier. Removing this modifier after performing our
mitigation results in an applicable update. It suggests that AppEvolve does
not support the {\tt final} modifier.

\subsection{Result assessment}

We were initially surprised by the large difference between the results on
our data set, where we obtained a success rate of 18.51\%, and the results
reported by the authors of AppEvolve, who obtained a success rate of
85.00\%, even though our dataset was constructed randomly, without taking
into account the strengths or weaknesses of AppEvolve.  Indeed, given that
AppEvolve relies on matching statements, abstracting only over variables,
it is surprising that AppEvolve could achieve such high accuracy on
e.g.\ simple getter methods such as {\tt getCurrentHour()} (e.g., Listing
3). In our dataset, such methods are often used as subexpressions of
arbitrarily complex statements, which may contain project-specific code.
Likewise, as illustrated by Listing 2, invocations of method calls may have
arbitrarily complex, possibly project-specific, arguments that are not
likely to be found in other codebases.

To understand better how AppEvolve is able to achieve a high rate of
success in the previously reported evaluation, we investigate the examples
and target apps used.  Given the difficulty of finding the original files
for the examples, for which no origin information is provided, we focus on
a single example, the call to {\tt getCurrentHour()} in the app {\sc
  Conversations} version 1.8.0 in F-Droid (A02-U03 in the AppEvolve paper).
In F-Droid, we find that the only invocation of {\tt getCurrentHour()}
occurs in the argument list of a call to the {\tt set} method of the {\tt
  Calendar} class, a code structure similar to the code shown on the left
side of Listing 3.  In the AppEvolve dataset, this call is extracted into a
separate statement, analogous to our mitigation shown on the right side of
Listing 3.  We have furthermore looked at one of the examples provided for
this API, and have found the original code on GitHub in the file {\tt
  RepeaterDialogFragment.java} from the {\tt
  marekpiotrowskimp/nyndro\_remote} repository.  Again, we find that in the
GitHub code, the call appears in an argument list, while the corresponding
example in the AppEvolve example set has been transformed analogous to our
mitigation.  These changes convert invocations that occur in differing
contexts to invocations that can be exploited by AppEvolve.

We note that the observed issue appears to be a direct result of the key
design decision of AppEvolve, of not merging examples.  Merging could
motivate discarding irrelevant context of the use of the deprecated method
and extending the abstraction of subterms beyond variables, but, as observed
by the authors of AppEvolve, it could discard some information
that is necessary to correctly perform the transformation.

\subsection{Future Directions}

Our above analysis suggests that our mitigation succeeds because the
AppEvolve authors have already performed such a mitigation in the change
examples included in the AppEvolve replication package.  To improve the
results of AppEvolve in the more general case, a solution could be to
systematically first normalize the change examples to a standard form.
Edits can then be learned from this standard form. When the edit is applied
to a new piece of code, that target code should also be normalized, to
minimize the code variations. If the edits are successfully applied, the
resulting code can then be refactored again to restore the coding style of
the developers.  Our mitigations represent a subset of the normalizations
that may be possible.

Any tool that automatically transforms code runs the risk of converting the
code to a style that is incompatible with the preferences of the given
app's developers.  In the case of AppEvolve, there is already the risk of
changing the existing coding style to the coding style found in the app
providing the change example, and any normalization performed to the target
code further increases this risk.  To assess the impact of AppEvolve on
coding style, we perform a simple experiment to compare the output of
AppEvolve after applying our mitigation with code that is manually updated
to deal with a set of deprecated APIs.  We asked a software engineer who is
not an author of this paper to update the code shown on the left hand side
of Listing 4 in Table~\ref{tab:mitigatesucc}. The code generated by
AppEvolve and by the engineer is shown below.

%Buse and Weimer~\cite{Buse:2008:MSR:1390630.1390647} propose a metric of
%software readability that are based on a set of simple local features. They
%also provide an automatic readability measure model that is built on 12,000
%code readability judgments. We run this model and compare the readability
%measure of the API update created by AppEvolve against that of the update
%created by the human.  Scores range from 0 to 1 where, 0 is least and 1 is
%most readable.

\vspace{0.2cm}\noindent {\small\bf AppEvolve's Update}\ft{Need to update this example}
\begin{lstlisting}[language=diff,numbers=none]
public void displayTime(View view){
  int varInt1;
  int varInt2;
  if (android.os.Build.VERSION.SDK_INT >=
      android.os.Build.VERSION_CODES.M) {
    varInt1=timePicker.getHour();
  }
  else {
    varInt1=timePicker.getCurrentHour();
  }
  if (android.os.Build.VERSION.SDK_INT >=
      android.os.Build.VERSION_CODES.M) {
    varInt2=timePicker.getMinute();
  }
  else {
    varInt2=timePicker.getCurrentMinute();
  }
  String time=varInt1 + ":" + varInt2;
  Toast.makeText(this,time,Toast.LENGTH_SHORT).show();
}
\end{lstlisting}


\vspace{0.2cm}\noindent {\small\bf Engineer's Update}
\begin{lstlisting}[language=diff,numbers=none]
public void displayTime(View view) {
    String time;
    if (android.os.Build.VERSION.SDK_INT >= 23) {
        // only for gingerbread and newer versions
        time = timePicker.getHour() + ":" +
               timePicker.getMinute();
    } else {
        time = timePicker.getCurrentHour() + ":" +
               timePicker.getCurrentMinute();
    }
    Toast.makeText(this, time, Toast.LENGTH_SHORT).show();
}
\end{lstlisting}

%\jl{the following two paragraphs have to be merged.}

\vspace{0.5cm}The code made manually by the engineer is substantially
different from the code produced by AppEvolve, and developers may indeed
prefer the code made by the engineer. In particular, the engineer did not
introduce fresh variables and introduced only one test of the Android
version, covering both method uses.  Future work may investigate how to
restore the original coding style after applying AppEvolve. One step would
be to choose good names for the newly introduced variables. Some recent
studies use deep learning and mining software repositories to recommend
method and class names~\cite{allamanis2015suggesting}. These approaches can
potentially be extended for inferring good variable names in our context.
