In recent years, each release of Android has seen the deprecation of some
APIs and the introduction of new ones.  Due to the fragmentation of the
Android market~\cite{he2018understanding,li2018cid}, app developers must
update their apps to use the new APIs, while maintaining backward
compatibility.  Unfortunately, updating API uses is not always trivial.
While API changes are described in the documentation, these descriptions
are not always accompanied by concrete examples.  When an API is used at
many places in the app source code, updating its uses is tedious and
error prone.

AppEvolve was recently proposed by Fazzini et
al.~\cite{fazzini2019automated} to automate the updating of Android apps in
response to deprecations of Android APIs.  The main idea of AppEvolve is
that many software projects may use the APIs in similar ways, and thus
existing codebases may already contain examples of how to update uses of
deprecated APIs.  Given a target app that uses a deprecated API and
information about the new API method(s) that should replace it, AppEvolve
collects update examples from existing codebases, abstracts their variable
references, ranks them according to their expected applicability, and then
tries to apply the abstracted examples one by one to the target app.  If an
abstracted example matches the code, the update is said to be {\em
applicable}.  Applicable updates are furthermore validated by testing.  A
key feature of AppEvolve is that it does not merge the abstracted examples
into a single generic pattern, to avoid losing information that may be
specific to particular usage contexts.

AppEvolve has been evaluated on a dataset involving 15 real-world apps and
20 real API changes. The dataset contains 41 usage locations of these real
 API changes. AppEvolve successfully produced validated applicable updates
for 85\% of these API changes and 90\% of their usage locations.  The
paper on AppEvolve is accompanied by a replication package, containing the
tool and the
dataset.\footnote{\url{https://sites.google.com/view/appevolve}} We have
reproduced the results of AppEvolve on this dataset.

In this work, our goal is to understand the generalizability of the results
reported for AppEvolve, and to analyze the characteristics of the apps for
which AppEvolve cannot generate applicable updates.  To this end, we
consider the same APIs and change examples as used in the AppEvolve
evaluation, but create a new dataset of target apps.  In selecting the new
target apps, we make no effort to take the expected strengths or weaknesses
of AppEvolve into account, and only check that the apps use one of the
deprecated APIs considered in the AppEvolve dataset and do not use the
corresponding replacement API.  Of the apps we found at GitHub that satisfy
these properties, we randomly selected 54 mobile apps that are entirely
distinct from the original data set, and that involve 14 of the 20 real API
changes and 54 usage locations.

We applied AppEvolve to update the deprecated-API usages in these mobile
apps and analyzed the cases where it fails to update the apps. Indeed,
running AppEvolve on our dataset generates applicable updates for only 10
out of 54 usage locations.  It generates only failed updates, i.e., updates
that do not match the deprecated-API invocation, for the remaining 44 usage
locations. Our analysis shows that AppEvolve produces applicable updates
for a mobile app in cases where the API usage closely matches an API change
example.  On the other hand, AppEvolve often fails to update a mobile app
when there are minor syntactical differences between the invocations of the
deprecated API in the target app and the change example.  It also fails
when there are semantic differences, such as the use of inheritance or the
use of {\tt static} or {\tt final} modifiers, when encountering edits that
require modifications beyond method boundaries, and when finding no
relevant change examples in GitHub.  Of these limitations, only the latter
two were mentioned in the AppEvolve paper.

To try to understand the issues leading to the drastic performance
difference as compared to the results of the original AppEvolve evaluation,
we refactor the use of the deprecated in the target app code so that it
more closely resembles the uses found in the change examples from which
AppEvolve learns the edits.  After refactoring, we found that AppEvolve can
generate applicable updates for 86\% of the cases where it previously
failed to do so.  Based on these results, we propose to mitigate the
dependence of AppEvolve on the code structure of the examples and the
deprecated API usage by normalizing the code to a standard form coupled
with the use of identifier-name recommendation, to improve existing work on
automated deprecated-API usage updates for Android apps.

The main contributions of this paper are as follows:
\begin{itemize}
	\item We  evaluate AppEvolve with additional Android apps to investigate if it can effectively perform API-usage updates in different situations. The dataset that we created contains three times more real-world apps than those used to evaluate AppEvolve in the original paper (54 vs 15).
	\item We categorize the limitations of AppEvolve and show API usage
	instances that are affected by these limitations.
	\item We discuss how the limitations may be overcome to improve the automatic API-usage updates for Android apps.
	\item We release a replication package to allow easy replication
	and evaluation of our experiments and results.\footnote{\url{https://sites.google.com/smu.edu.sg/appevolve-replication}}
	\lx{if it's really 30GB, better to explain a bit why it is so big,
	e.g., containing much raw data crawled from GitHub?} \jl{This could
	be in the footnote.}
\end{itemize}

The remainder of this paper is structured as follows. Section~\ref{sec:approach} describes AppEvolve, the state-of-the-art work on automated updates of deprecated API usages in Android apps. Section~\ref{sec:replication} presents our replication settings. Section~\ref{sec:findings} presents our findings. Section~\ref{sec:mitigations} presents our mitigation strategy and results for some failed updates of Android apps.
Section~\ref{sec:discuss} discusses limitations and threats to validity of our findings.
Section~\ref{sec:related} presents related work.
Finally, Section~\ref{sec:conclusion} concludes and presents promising future work.
