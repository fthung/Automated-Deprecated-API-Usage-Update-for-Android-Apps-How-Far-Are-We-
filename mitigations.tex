To mitigate the observed failures, we tried to modify the deprecated API
usages in the target app's source code so that AppEvolve could produce an
applicable update. Looking at the 10 apps that were updated successfully,
we observe that in each case the invocation of the deprecated method
appears as the right hand side of an assignment statement, or any
arguments of the deprecated method are simple variables.
We transform the remaining apps where possible accordingly, converting the
invocations of the deprecated APIs to a form reminiscent of the three
address code used in compiler intermediate representations \cite{dragon}.
The right side of Table~\ref{tab:mitigatesucc} shows a diff that applies
these mitigations on the cases presented in Section~\ref{sec:findings}
where the mitigation allows AppEvolve to produce applicable updates.

\begin{enumerate}
\item After the refactoring described in Listing 1
Table~\ref{tab:mitigatesucc}, rather than directly returning the result of
invoking the deprecated method {\tt fromHtml}, we first assign it to a fresh
variable named {\tt a} of type {\tt Spanned}.

\item After the refactoring described in Listing 2 of
Table~\ref{tab:mitigatesucc}, rather than directly inserting the constants
{\tt AudioManager.STREAM\_MUSIC} and {\tt AudioManager.AUDIOFOCUS\_GAIN} as
the second and third arguments of {\tt requestAudioFocus}, we first assign
them to fresh {\tt int}-typed variables {\tt arg1} and {\tt arg2},
respectively. Moreover, the result of invoking {\tt requestAudioFocus} is
assigned to a fresh variable named {\tt res} of type {\tt AudioManager}
that is then used in the if condition.

\item After the refactoring described in Listing 3 of
Table~\ref{tab:mitigatesucc}, rather than directly inserting the result of
invoking {\tt getCurrentHour} as the second argument of {\tt
String.format}, the result is first assigned to fresh a variable named {\tt hour}
of type {\tt int}.

\item After the refactoring described in Listing 4 of
Table~\ref{tab:mitigatesucc}, rather than concatenating the result of {\tt
getCurrentHour} directly to ``:'', the result is first assigned to a fresh
variable named {\tt hour} of type {\tt int}.

\item After the refactoring described in Listing 5 of
Table~\ref{tab:mitigatesucc}, rather than directly assigning the result of
invoking {\tt getCurrentHour} to the variable {\tt hours} when it is
declared, we declare {\tt hours} first and then assign the result of
invoking {\tt getCurrentHour} to {\tt hours}.

\item After the refactoring described in Listing 6 of
Table~\ref{tab:mitigatesucc}, rather than invoking {\tt setTextAppearance}
directly from the object returned by invoking {\tt findViewById}, the
returned object is first assigned to a variable named {\tt t} of type {\tt
TextView}. {\tt setTextAppearance} is then invoked from {\tt t}.

\end{enumerate}

In essence, our mitigations refactor the deprecated method invocation in
the target app to resemble the examples from which AppEvolve learns
edits. Our mitigations result in 38 successful updates out of the 44 failed
updates on the apps in our dataset (86\% success rate), overall giving
success on 48 out of the 54 apps in our dataset (89\% success rate).
