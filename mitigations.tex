We tried to modify the target location in the app's source code so that
AppEvolve can produce an applicable update. For most cases that AppEvolve
fails to update the apps, we notice that simple refactoring of the target
app might suffice to allow edits learned from examples to be applied
successfully in the target app. We show these successful modifications on
cases presented in Section~\ref{sec:findings}. In each case, we apply a
mitigation with respect to a specific use of a specific deprecated API
method, and then run AppEvolve on the code resulting from applying the
mitigation.

\begin{enumerate}
\item We refactor the code in the left hand of Listing 1 in Table~\ref{tab:mitigatesucc} into the code in its right hand side. In this refactored code, rather than directly returning the result of {\tt fromHtml} deprecated method invocation, we first assign it to a variable named {\tt a} of type {\tt Spanned}. %This allows AppEvolve to generate an applicable update to this piece of code.

\item We refactor the code in the left hand of Listing 2 in Table~\ref{tab:mitigatesucc} into the code in its right hand side. In this refactored code, rather than directly inserting {\tt AudioManager.STREAM\_MUSIC} and {\tt AudioManager.AUDIOFOCUS\_GAIN} constants as the second and third arguments of {\tt requestAudioFocus} deprecated method invocation, they are first assigned to variables named {\tt arg1} of type {\tt int} and {\tt arg2} of type {\tt int}, respectively. Moreover, the result of {\tt requestAudioFocus} deprecated method invocation is assigned to a variable  named {\tt res} of type {\tt AudioManager}, which is then used in an if condition.

\item We refactor the code in the left hand of Listing 3 in  Table~\ref{tab:mitigatesucc} into the code in its right hand side. In this refactored code, rather than directly inserting the result of invoking {\tt getCurrentHour} deprecated method to {\tt String.format} second argument, the result is first assigned to a variable named {\tt hour} of type {\tt int}.

\item We refactor the code in the left hand of Listing 4 in  Table~\ref{tab:mitigatesucc} into the code in its right hand side. In this refactored code,, rather than concatenating the result of invoking {\tt getCurrentHour} deprecated method directly with ":" string, the result is first assigned to a variable named {\tt hour} of type {\tt int}.

\item We refactor the code in the left hand of Listing 5 in  Table~\ref{tab:mitigatesucc} into the code in its right hand side. In this refactored code, rather than directly assigning the result of invoking {\tt getCurrentHour} to a variable named {\tt hours} when it is declared, we declare the variable {\tt hours} first and then assign the result of invoking {\tt getCurrentHour} to the variable {\tt hours}.

\item We refactor the code in the left hand of Listing 5 in  Table~\ref{tab:mitigatesucc} into the code in its right hand side. In this refactored code, rather than invoking {\tt setTextAppearance} deprecated method directly from the object returned by invoking {\tt findViewById} method, the returned object is first assigned to a variable named {\tt t} of type {\tt TextView}. {\tt setTextAppearance} method is then invoked from the variable {\tt t}.

\end{enumerate}

As illustrated above for each suitable category, our mitigations result in 36 out of 44 updates that were failed to be applicable, which covers 81.82\% of the failed updates. In essence, our mitigations refactor the deprecated method invocation in the target app to resembles the example from which AppEvolve learns the edits to be applied to the target app. 
